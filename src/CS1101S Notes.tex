\documentclass[11pt]{article}
\usepackage[utf8]{inputenc}	% Para caracteres en español
\usepackage{amsmath,amsthm,amsfonts,amssymb,amscd}
\usepackage{multirow,booktabs}
\usepackage[table]{xcolor}
\usepackage{xcolor}
\usepackage{fullpage}
\usepackage{lastpage}
\usepackage{enumitem}
\usepackage{fancyhdr}
\usepackage{mathrsfs}
\usepackage{wrapfig}
\usepackage{setspace}
\usepackage{calc}
\usepackage{multicol}
\usepackage{cancel}
\usepackage[retainorgcmds]{IEEEtrantools}
\usepackage[margin=3cm]{geometry}
\usepackage{amsmath}
\newlength{\tabcont}
\setlength{\parindent}{0.0in}
\setlength{\parskip}{0.05in}
\usepackage{empheq}
\usepackage{framed}
\usepackage[T1]{fontenc}
\usepackage{tgbonum}
\usepackage[most]{tcolorbox}
\usepackage{xcolor}
\colorlet{shadecolor}{orange!15}
\parindent 0in
\parskip 12pt
\geometry{margin=1in, headsep=0.25in}
\theoremstyle{definition}
\newtheorem{defn}{Definition}
\newtheorem{reg}{Rule}
\newtheorem{exer}{Exercise}
\newtheorem{note}{Note}
\begin{document}
\renewcommand{\labelenumii}{\arabic{enumi}.\arabic{enumii}}
\renewcommand{\labelenumiii}{\arabic{enumi}.\arabic{enumii}.\arabic{enumiii}}
\renewcommand{\labelenumiv}{\arabic{enumi}.\arabic{enumii}.\arabic{enumiii}.\arabic{enumiv}}
\setcounter{section}{0}
\title{Chapter 9 Review Notes}
\newcommand{\code}{\fontfamily{pcr}\selectfont}
\thispagestyle{empty}

\begin{center}
{\LARGE \bf CS1101S Notes}\\
Prepared by Devansh Shah (Last updated on 4 Dec 2021 by Devansh Shah)
\end{center}
\tableofcontents
\pagebreak
\section{General Tips}
\begin{enumerate}
    \item When understanding a program, think from the highest possible level of abstraction. If you have truly understood what a function does, you should be able to explain it clearly and succinctly in a single sentence. Don't get bogged down by the details.
    \item Avoid complicating your programs unnecessarily. Elegance lies in simplicity. Make use of existing abstractions like map, filter, accumulate, build\_list, enum\_list etc.
    \item Perfection is finally attained not when there is no longer anything to add, but when there is no longer anything to take away - Antoine de Saint-Exupery
    \item Use wishful thinking and recursion wherever possible. Ask yourself: If I had a function that could solve a smaller sub-problem, how would I use that to solve this problem?
    \item There are two steps to programming a solution to a problem - solving the problem and writing the program. They must be done in that sequence. Solve the problem on paper before attempting to write the code. Don't even think of programming if you've no clue what to do
    \item Think carefully while programming to avoid making careless mistakes. It's easy to make mistakes while typing if you don't stay alert and pay attention to what you're typing.
    \item Use comments judiciously - explain the logic of the program but there's no need to explain each line if it is obvious.
\end{enumerate}
\section{Important Programs}

\subsection{List Processing}
\begin{enumerate}
    \item Permutations of a list
    \begin{shaded}
    {\code
    \text{function permutations(xs) \{} \\
    \text{\qquad if (is\_null(xs)) \{} \\
    \text{\qquad \qquad return list(null);} \\
    \text{\qquad \} else \{ }\\
    \text{\qquad \qquad return accumulate(append,null,} \\
    \text{\qquad \qquad \qquad map(x =$>$ } \\
    \text{\qquad \qquad \qquad \qquad \qquad map(y =$>$ pair(x,y),permutations(remove(x,xs))),} \\
    \text{\qquad\qquad\qquad\qquad xs);}\\
    \text{\qquad \}} \\ 
    \}
    }
    \end{shaded}
    \item Choosing k elements from a list
    \begin{shaded}
    {\code
        \text{function combinations(xs,k) \{} \\
        \text{\qquad if (k === 0) \{} \\
        \text{\qquad \qquad return list(null);} \\        \text{\qquad \} else if (is\_null(xs)) \{ }\\
        \text{\qquad \qquad return null;} \\
        \text{\qquad \} else \{ } \\
        \text{\qquad \qquad const A = combinations(xs,k); } \\
        \text{\qquad \qquad const B = combinations(xs,k-1); } \\
        \text{\qquad \qquad const C = map(x =$>$ pair(head(xs),x), B); } \\
        \text{\qquad \qquad return append(A,C); } \\
        \text{\qquad \}} \\
        \text{ \} }
       }
    \end{shaded}
    \item Subsets of a list
    \begin{shaded} 
    {\code
        \text{function subsets(xs) \{} \\
        \text{\qquad if (is\_null(xs)) \{} \\
        \text{\qquad \qquad return list(null);} \\
        \text{\qquad \} else \{ } \\
        \text{\qquad \qquad const A = subsets(tail(xs)); }
        \\
        \text{\qquad \qquad const B = map(x =$>$ pair(head(xs),x), A); } \\
        \text{\qquad \qquad return append(A,B); } \\
        \text{\qquad \}} \\
        \text{ \} }
        }
        \end{shaded}
    \item Mapping over a tree
    \begin{shaded} 
    {\code
    \text{function tree\_map(f,tree) \{ } \\ 
    \text{\qquad if (is\_null(tree)) \{ } \\
    \text{\qquad\qquad return null; } \\
    \text{\qquad \} else \{ } \\
    \text{\qquad\qquad return !is\_pair(head(tree)) } \\
    \text{\qquad\qquad\qquad ? pair(f(head(tree)),tree\_map(f,tail(tree))} \\
    \text{\qquad\qquad\qquad : pair(tree\_map(f,head(tree)),tree\_map(f,tail(tree)));} \\
    \text{\qquad \} } \\
    \text{ \} }
    }
    \end{shaded}
\end{enumerate}
\subsection{Stream Processing}
\begin{enumerate}
    \item Interweaving two infinite streams
    \begin{shaded} 
    {\code
    \text{function interweave(s1,s2) \{ } \\
    \text{\qquad return pair(head(s1), () =$>$ interweave(s2,stream\_tail(s1));} \\
    \text{\} }
    }
    \end{shaded}
    \item Sieve of Eratosthenes (Stream of Primes)
    \begin{shaded} 
    {\code
    \text{function sieve(s) \{ } \\
    \text{\qquad return pair(head(s),} \\
    \text{\qquad \qquad () =$>$ sieve(stream\_filter(x=$>$x\%head(s) !== 0,stream\_tail(s))));} \\
    \text{\} } \\
    \text{const primes = sieve(integers\_from(2));}
    }
    \end{shaded}
\end{enumerate}
\subsection{Array Processing}
\begin{enumerate}
    \item Insertion sort for Arrays
    \begin{shaded} 
    {\code
        \text{function insertion\_sort(A) \{ } \\
        \text{\qquad const len = array\_length(A);}\\
        \text{\qquad for (let i = 0; i $<$ len; i = i + 1) \{} \\
        \text{\qquad \qquad const x = A[i];} \\
        \text{\qquad \qquad let j = i - 1;} \\
        \text{\qquad \qquad while (j $>$= 0 \&\& A[j] $>$ x) \{ } \\
        \text{\qquad \qquad \qquad A[j+1] = A[j]; } \\
        \text{\qquad \qquad \qquad j = j - 1; } \\
        \text{\qquad \qquad \} } \\
        \text{\qquad \qquad A[j +1] = x; } \\
        \text{\qquad \} } \\
        \text{ \} }
        }
        \end{shaded}
    \item Reversing an array in place
    \begin{shaded} 
    {\code
        \text{function reverse(A) \{ } \\
        \text{\qquad const len = array\_length(A);}\\
        \text{\qquad for (let i = 0; i $<$ len/2; i = i + 1) \{} \\
        \text{\qquad \qquad const temp = A[i];} \\
        \text{\qquad \qquad A[i] = A[len-i-1];} \\
        \text{\qquad \qquad A[len-i-1] = temp;} \\
        \text{\qquad \} } \\
        \text{ \} } 
        }
        \end{shaded}
    \item 
\end{enumerate}
\subsection{Miscellaneous}
\begin{enumerate}
    \item N-choose-k algorithm
    \begin{shaded} 
    {\code
        \text{function choose(n,k) \{ } \\
        \text{\qquad return n $<$ 0 || n $<$ k } \\
        \text{\qquad \qquad ? 0 } \\
        \text{\qquad \qquad : n === k} \\
        \text{\qquad \qquad ? 1 } \\
        \text{\qquad \qquad : choose(n-1,k) + choose(n-1,k-1);} \\
        \text{ \} } 
        }
        \end{shaded}
    \item Implementing accumulate using continuation passing style (CPS)
    \begin{shaded} 
    {\code
        \text{function accumulate\_cps(f,init,xs) \{ } \\
        \text{\qquad function acc(ys,cont) \{ } \\
        \text{\qquad \qquad return is\_null(ys) } \\
        \text{\qquad \qquad \qquad  ? cont(init) } \\
        \text{\qquad \qquad \qquad: acc(tail(ys), x $=>$ cont(f(head(ys),x)));} \\
        \text{\qquad \} } \\
        \text{\qquad return acc(xs, x $=>$ x); } \\
        \text{ \} } 
        }
        \end{shaded}
    \item 
\end{enumerate}
\section{Important Concepts}
\subsection{Identity: ===}
\begin{enumerate}
\item {\color{red}True, false, null, undefined} - each is identical to itself and nothing else.
\item Two {\color{red}numbers} are identical if they have the same representation in the double precision floating point representation.
\item Two {\color{red}strings} are identical if they have the same characters in the same order.
\item {\color{red}Functions} are made by function expressions, and their creation bestows an identity upon them. 
\item {\color{red} Pairs} are made by the pair function, and their creation bestows an identity upon them.
\end{enumerate}
\subsection{Environment Model}
\begin{enumerate}
    \item Compound objects like user-declared functions, pairs, arrays, etc. are drawn outside the frames. Primitive data like numbers, strings, booleans, etc. are written inside the frame.
    \item No empty frame is created.
    \item To evaluate a function declaration, two circles are drawn - the left one points to the text of the function body (and parameters), while the right one points to the frame in which the function was declared.
    \item To evaluate a function application, the function expression and arguments are evaluated in the current environment, a parameter frame is created that extends the function's environment, and the body block of a function is evaluated in a new frame that extends the parameter frame.
    \item Each time the body of a while loop is run, a new frame is created if there are constant declarations made in the body.
    
\end{enumerate}
\subsection{Metacircular Evaluator}
\begin{enumerate}
    \item {\code evaluate(component, environment)} - evaluates the component with respect to the given environment. A component is either a statement or an expression.
    \item {\code apply(fun, args)} - if the function is primitive, applies the function to the arguments. Otherwise, it evaluates the function body by extending the function's environment.
    \item {\code list\_of\_values(args)} - evaluates each of the arguments passed to the function (before it is passed to apply, obeying applicative order reduction). 
    \item {\code extend\_environment(symbols,vals,base\_env)} - Creates a new frame by extending the base environment, which contains a list of symbols and a list of associated values.
    \item We represent an environment as a list of frames. Each frame is a pair of lists - the first list contains the names of the declarations in that frame, and the second list contains the values associated with the names declared in that frame (in the same order)
    \item Operator combinations are converted to function applications to be evaluated. Function declarations are converted to constant declarations whose values are lambda expressions.
    \item Our MCE cannot handle arrays, for loops, while loops (although we discussed how to evaluate while loops and for loops), break and continue statements.
    \item Our MCE does not distinguish between variables and constants. In particular, it allows assignment to constants, quite erroneously.
    \item Our MCE does not detect undeclared names in statements that are not evaluated (for example, if the consequent of a conditional contains an undeclared name but the predicate evaluates to false, our MCE will not complain). The Source Academy, on the other hand, gives error messages for any name that is not declared in the program.
    \item Our MCE does not support logical composition operators like {\code \&\&} and {\code ||}. We can easily modify this by converting them to conditional expressions of the form \\ {\code predicate ? consequent : alternative}. (Recall that {\code x \&\& y} is identical to {\code x ? y : false}, and {\code x || y} is identical to {\code x ? true : y}.
    \item When we transform our MCE to lazily evaluate the program (normal order reduction), we have to create objects called "thunks", which represent the expressions that have not been evaluated. We need to keep track of the environment in which these thunks need to be evaluated at some point. Once they have been evaluated, we can choose to memoize the result so that we don't need to evaluate the thunks again.
    
\end{enumerate}
\section{Some other Useful Facts}
\begin{enumerate}
    \item Source does not hoist functions to the top of the block in which they are declared. This is because function declaration is treated identical to a constant declaration.
    \item All pre-declared list processing functions give rise to iterative processes since they are implemented using Continuation Passing Style.
    \item In JavaScript, there is a critical difference between for loops and while loops: In case of for loops, each iteration of the body has its own "copy" of the loop control variable, which is not the case in while loops.
    \item The {\code equal(x,y)} function does not use === to check for equality of pairs. It returns true if both the arguments have the same structure with respect to pair, and identical values at corresponding leaf positions (that are not themselves pairs). Strings, numbers, boolean, undefined, null etc. are directly compared using ===. Arrays cannot be compared using {\code equal(x,y)}.
    \item {\code member(v,xs)} uses === to check if v occurs in the list xs. It does not use {\code equal(x,y)}.
    \item Assignment and look-up takes constant time for arrays. Access of an array with an array index to which no prior assignment has been made on the array
    returns {\code undefined}.
    \item Some stream functions are partially lazy. For example, {\code stream\_filter(pred,s) } is lazy only after it has found the first element in the stream that satisfies the predicate.
\end{enumerate}
\end{document}
