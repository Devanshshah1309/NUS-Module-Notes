\documentclass[11pt]{article}
\usepackage[utf8]{inputenc}	% Para caracteres en español
\usepackage{amsmath,amsthm,amsfonts,amssymb,amscd}
\usepackage{multirow,booktabs}
\usepackage[table]{xcolor}
\usepackage{xcolor}
\usepackage{fullpage}
\usepackage{lastpage}
\usepackage{enumitem}
\usepackage{fancyhdr}
\usepackage{mathrsfs}
\usepackage{wrapfig}
\usepackage{setspace}
\usepackage{calc}
\usepackage{multicol}
\usepackage{cancel}
\usepackage[retainorgcmds]{IEEEtrantools}
\usepackage[margin=3cm]{geometry}
\usepackage{amsmath}
\newlength{\tabcont}
\setlength{\parindent}{0.0in}
\setlength{\parskip}{0.05in}
\usepackage{empheq}
\usepackage{framed}
\usepackage[T1]{fontenc}
\usepackage{tgbonum}
\usepackage[most]{tcolorbox}
\usepackage{xcolor}
\colorlet{shadecolor}{orange!15}
\parindent 0in
\parskip 12pt
\geometry{margin=1in, headsep=0.25in}
\theoremstyle{definition}
\newtheorem{defn}{Definition}
\newtheorem{reg}{Rule}
\newtheorem{exer}{Exercise}
\newtheorem{note}{Note}
\begin{document}
\renewcommand{\labelenumii}{\arabic{enumi}.\arabic{enumii}}
\renewcommand{\labelenumiii}{\arabic{enumi}.\arabic{enumii}.\arabic{enumiii}}
\renewcommand{\labelenumiv}{\arabic{enumi}.\arabic{enumii}.\arabic{enumiii}.\arabic{enumiv}}
\setcounter{section}{0}
\title{Chapter 9 Review Notes}
\newcommand{\code}{\fontfamily{pcr}\selectfont}
\thispagestyle{empty}

\begin{center}
{\LARGE \bf Basic Terminal Commands}\\
Prepared by Devansh Shah (Last updated on 23 Dec 2021)
\end{center}
\begin{enumerate}
    \item {\code ls} - list all your files in the current directory
    \item {\code pwd} - print working directory
    \item {\code cd} - change directory. Example: {\code cd Documents} (the directory must be a folder in the working directory)
    \item {\code ..} - refers to parent directory
    \item {\code cd ..} - go back to parent directory
    \item {\code clear} - clears terminal
    \item {\code control+c} - kills any command running
    \item {\code touch [filename]} - creates a new file
    \item {\code vim [filename]} - opens file in vim for editting purposes
    \item {\code history} - used to see history of commands executed in terminal
    \item {\code open [filename]} - used to open file
    \item {\code rm [filename]} - remove file
    \item {\code mkdir [folder]} - create a new folder/directory 
    \item {\code rmdir [folder]} -  remove folder/directory
    \item {\code rm -r [folder]} - alternative way to remove folder (where -r refers to recursive, as it recursively deletes all files in the folder
    \item {\code cat [filename]} - prints contents of file
    \item {\code mv [file location] [new location]} - moves file to new location
    \item {\code [click on tab]} - auto-completes filename or directory name (if there is no ambiguity)
    \item {\code exit} - exits terminal
\end{enumerate}
\end{document}